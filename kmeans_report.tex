\documentclass[12pt, a4paper]{article}
\usepackage{ctex}  % 支持中文
\usepackage{graphicx}  % 插入图片
\usepackage{amsmath}  % 数学公式
\usepackage{listings}  % 代码列表
\usepackage{xcolor}  % 颜色
\usepackage{hyperref}  % 超链接
\usepackage{geometry}  % 页面设置
\usepackage{float}  % 图片浮动位置

% 页面设置
\geometry{left=2.5cm,right=2.5cm,top=2.5cm,bottom=2.5cm}

% 代码样式设置
\lstset{
    language=Python,
    basicstyle=\small\ttfamily,
    numbers=left,
    numberstyle=\tiny,
    keywordstyle=\color{blue},
    commentstyle=\color{green!60!black},
    stringstyle=\color{red},
    showstringspaces=false,
    breaklines=true,
    frame=single,
    captionpos=b
}

\title{基于K-means算法的鸢尾花数据集聚类分析}
\author{数据挖掘实验报告}
\date{\today}

\begin{document}

\maketitle

\begin{abstract}
本实验基于经典的鸢尾花(Iris)数据集,实现并分析了K-means聚类算法。通过对算法的实现、参数优化和结果可视化,深入研究了K-means算法在实际数据集上的应用效果。实验结果表明��K-means算法能够有效地对鸢尾花数据进行聚类分析,并取得了良好的聚类效果。
\end{abstract}

\tableofcontents

\section{引言}
\subsection{研究背景}
聚类分析是数据挖掘和机器学习中的重要任务之一,其目的是将相似的数据对象划分到同一个簇中,而将不相似的对象划分到不同簇中。K-means算法作为最经典的聚类算法之一,因其简单、高效的特点而被广泛应用。

\subsection{研究目的}
本实验旨在:
\begin{itemize}
    \item 实现K-means聚类算法
    \item 分析算法在鸢尾花数据集上的表现
    \item 通过可视化手段理解聚类结果
    \item 评估聚类效果并进行参数优化
\end{itemize}

\section{理论基础}
\subsection{K-means算法原理}
K-means算法的基本思想是通过迭代方式寻找K个簇的一种划分方案,使得聚类结果对应的代价函数最小。其主要步骤如下:

\begin{enumerate}
    \item 随机选择K个点作为初始聚类中心
    \item 计算每个数据点到各个聚类中心的距离,将其划分到最近的聚类中心所对应的簇
    \item 重新计算每个簇的中心点(计算簇内所有点的均值)
    \item 重复步骤2和3,直到聚类中心不再发生变化或达到最大迭代次数
\end{enumerate}

\subsection{评估指标}
本实验使用以下指标评估聚类效果:
\begin{itemize}
    \item SSE(簇内误差平方和):评估簇内的紧密度
    \item 轮廓系数:评估簇的分离度和紧密度
    \item 聚类准确率:与真实标签比较的正确率
\end{itemize}

\section{实验设计}
\subsection{数据集说明}
鸢尾花数据集包含150个样本,每个样本有4个特征:
\begin{itemize}
    \item 萼片长度(Sepal Length)
    \item 萼片宽度(Sepal Width)
    \item 花瓣长度(Petal Length)
    \item 花瓣宽度(Petal Width)
\end{itemize}

数据集包含三个品种的鸢尾花:Setosa、Versicolor和Virginica,每类50个样本。

\subsection{实验环境}
\begin{itemize}
    \item 编程语言:Python 3.6+
    \item 主要库:NumPy, Pandas, Matplotlib, Scikit-learn
    \item 开发环境:Visual Studio Code
\end{itemize}

\subsection{实验流程}
\begin{enumerate}
    \item 数据预处理
    \begin{itemize}
        \item 数据加载和清洗
        \item 特征标准化
        \item 数据可视化分析
    \end{itemize}
    
    \item 算法实现
    \begin{itemize}
        \item 实现K-means核心算法
        \item 实现评估指标计算
        \item 实现结果可视化
    \end{itemize}
    
    \item 参数优化
    \begin{itemize}
        \item 使用肘部法则确定最优K值
        \item 分析不同K值对聚类效果的影响
    \end{itemize}
\end{enumerate}

\section{实验结果与分析}
\subsection{数据分布分析}
通过特征分布图和散点矩阵,我们可以观察到:
\begin{itemize}
    \item 不同特征的分布特征
    \item 特征之间的相关性
    \item 数据的可分性
\end{itemize}

\subsection{聚类结果分析}
\subsubsection{最优K值选择}
通过肘部曲线和轮廓系数分析,我们发现:
\begin{itemize}
    \item K=3时,SSE下降趋势变缓
    \item K=3时,轮廓系数达到较高值
    \item 这与数据集中实际的三个品种数量相符
\end{itemize}

\subsubsection{聚类效果评估}
在K=3的情况下:
\begin{itemize}
    \item 聚类准确率达到较高水平
    \item 簇内紧密度良好
    \item 簇间分离度明显
\end{itemize}

\subsection{算法性能分析}
\begin{itemize}
    \item 收敛速度快,一般在10次迭代内收敛
    \item 计算复杂度适中,适合中等规模数据集
    \item 对初始中心点的选择较为敏感
\end{itemize}

\section{结论与展望}
\subsection{主要结论}
\begin{itemize}
    \item K-means算法在鸢尾花数据集上表现���好
    \item 算法能够有效识别数据集中的自然聚类
    \item 通过参数优化可以进一步提升聚类效果
\end{itemize}

\subsection{改进方向}
\begin{itemize}
    \item 优化初始中心点的选择策略
    \item 考虑使用其他距离度量方式
    \item 结合其他算法进行集成学习
\end{itemize}

\section{参考文献}
\begin{enumerate}
    \item MacQueen, J. (1967). Some methods for classification and analysis of multivariate observations.
    \item Fisher, R. A. (1936). The use of multiple measurements in taxonomic problems.
    \item Scikit-learn: Machine Learning in Python, Pedregosa et al., JMLR 12, pp. 2825-2830, 2011.
\end{enumerate}

\appendix
\section{代码实现}
\subsection{K-means核心实现}
\lstinputlisting[caption=K-means算法实现]{kmeans_improved.py}

\subsection{数据可视化实现}
\lstinputlisting[caption=数据可视化实现]{data_visualization.py}

\end{document} 